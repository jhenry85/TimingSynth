%\url{http://phys.org/news/2016-11-scientists-hackers-remotely-cars.html}

As CPS are resource-constrained systems, understanding the impact of any security solutions on  control performance, timing, and resources of the system is important. Furthermore, ensuring the solutions respect the semantic gap between design and implementation is crucial for its correct operation. Consequently, Lin \textit{et al.} \cite{LZV14}, Pasqualetti and Zhu \cite{PZ15} and Zheng \textit{et al.} \cite{ZDRP16} proposed frameworks that analyses the impact of security solutions and consider the gap between controller design and implementation. Lin \textit{et al.} \cite{LZV14} analysed the impact of message authentication mechanism with time-delayed release of keys on real-time constraints of a ground-vehicle. Such a security solution was developed to protect Time Division Multiple Access (TDMA)-based protocol, which is used in many safety-critical systems such as automobile and avionics electronic systems because of their more predictable timing behavior. To ensure the increased latencies (due to delayed key release) did not violate timing requirements, an algorithm to optimize task allocation, priority assignment, network scheduling, and key-release interval length during the mapping process from the functional model to the architectural model, with consideration of the overhead was developed. This algorithm combined simulated annealing with a set of efficient optimization heuristics. However, their approach did not consider the impact of their security solution on sampling periods and control performance. Furthermore, they didn't consider presence of a software platform between the security solution and hardware.     


Pasqualetti and Zhu's \cite{PZ15} method could analyse control performance, security, and end-to-end timing of a resource-constrained CPS under network (cyber) attack that can compromise systems privacy (confidentiality). They have also quantified interdependency between the three system objectives by means of a minimal set of interface variables and relations. In their work, they have considered an adversary that has complete knowledge of the system model and can reconstruct system states from measurements. As a first step, the physical plant was modeled as a continuous time LTI system. The control input was determined using an output-based control law. A relationship was established to show that the control performance improved with reduced sampling time. Next, resiliency of the encryption method, protecting messages transmitted by sensor to controller was evaluated. It was observed that the encryption method increased the sampling period thereby degrading control performance. While implementing the control function on a CPS platform, the end-to-end delay was calculated by incorporating time incurred during sensing, computation, and communication. During development of the scheduling algorithm for the system, it was ensured that the measured delay was within the sampling period. Based on their analysis, they concluded that the control and the security algorithms should be designed based on the implementation platform so as to optimize performance and robustness. 

Zheng \textit{et al.} \cite{ZDRP16} quantify the impact of their security solution on control performance and schedulability. They also analyzed the tradeoffs between security level and control performance while ensuring the resource and real-time constraints were satisfied. For demonstration, a CPS with multiple controllers that share computation and communication resources was considered. A controller, which was modelled as a control task, processed information collected from sensors on a shared resource and commanded actuators to carry out task. To prevent attackers from eavesdropping on the communication medium for obtaining system's internal information, messages from sensors were encrypted. The decryption of these messages were modeled as task. Each of these tasks were given an activation period and worst case execution time. In the system, the control tasks competed for computation resources whereas as messages competed for communication resources. Incorporating the message encryption mechanism introduced resource overhead that impacted schedulability and control performance. To avoid this issue, they framed an optimization problem where control performance (a function of control task period) was the objective function and security level, computation resource, communication resource, and end-to-end latency were constraints. By varying the security level (function of messages to be encrypted), they ensured that the system achieved optimal control performance and platform schedulability. 

