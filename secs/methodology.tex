
\textbf{Goal of Project}

Given a CPS platform and application SW, how can we find timing bounds which are extensible i.e. evolution or modification of the system doesn't effect the timing bounds. 

A key assumption in verifying time constraints on  embedded cyber physical systems is the that the given timing model is correct.
The time for each step between nodes of the automata are given upper and lower timing bounds - usually given by the manufacturer of the device.
This expands the trusted base to include the manufacturer.

While manufacturer guarantees can often be taken as safe assumptions, such models are not available in many other situations.
For example, when embedding platform independent software into a particular system, the timing model may change based on this hardware.
Furthermore, as cyber physical system component development becomes more accessible to individuals, there may not be a central manufacturer that can provide a bounded timing model.

\section{Motivations}

1) cant trust the manufacture

2) the manufacturer things could be hacked, and we want to know 'how can the timing guarantees from the manufacture be hacked' so that we still have a safe/stable system.


\textbf{Outline of approach}

\begin{enumerate}
    \item Derive average timing bounds for each atomic step of the CPS through simulation.
    \item To safely generalize the simulation Assume a probabilistic distribution over the simulation time to get bounds.
    \item Using a model checker, find the minimal distribution that implies bounds on the timing model such that the system still satisfies the safety conditions.
\end{enumerate}


\iffalse

http://adt.cs.upb.de/timmo-2-use/publications.htm

http://ieeexplore.ieee.org/abstract/document/7927248/

http://dl.acm.org/citation.cfm?id=2883827

http://jin.ece.ufl.edu/papers/ISVLSI16_CPS.pdf

\fi
% How are timing model/ constraints derived in vehicles? 

































%\textit{Informal Definition of Problem:}

%\begin{itemize}
%    \item 1. Impact of 
%\end{itemize}

% Understanding the impact of security solutions on performance, timing, and resources of the system is essential for its effective operation. Effecting the performance of a real-time resource-constrained CPS could result in undesired consequence. As such we intend to develop a framework that can analyze the impact of security solutions on design metrics of the system. In our framework, we model the system architecture of CPS using platform-based design methodology \cite{SDP12}. The design requirements such as performance, timing, resource along with security are represented as \textit{contracts}. A \textit{contract} is a pair $\mathcal{C} = (A, G)$ of $\{$Assumptions, Guarantee$\}$ of properties that must be satisfied by the set of all inputs and all outputs to the system. We use \textit{quantitative model checking} technique to verify the system against these contracts \cite{GS04}. In this approach, a model $\mathcal{M}$ of a system with an initial state \verb"s"$_{0}$ is expressed as a discrete or continuous time Markov chain, contracts, $\varphi$, are represented using a specification language such as Linear Temporal Logic. Using these notations, the model checking problem can be formally stated as $\mathcal{M},$ \verb"s"$_0 \models \varphi $. %\sim p$ iff $\mathbb{P}_{\mathcal{M}^*} (\underset{\sigma \in Act}{\bigcup} \pi (s_0, \sigma o_\varphi)) \sim p$ .
% %, $p \in [0,1]$ is a probability threshold, and $\bowtie \, \in \{ > , < , \geq, \leq \}$.
% While verifying the system, it should be ensured that the contracts are satisfied within some user defined probability bound. Based on results of the verification process, we can quantify the impact of the design metrics on the system. If the system satisfies all the contracts, we can \textit{certify} the security solution to be appropriate for the system.

% New Ideas

% Resource managment while under attack

% in the case of V2V communications, can the bandwidth be overloaded?
% If a resource is being over utilized in a way consistent with an attack, do a \textit{soft restart} of that resource.
% Wipe the memory, but not the whole system, in order to return to a safe uncompromized state. 
% Need to then show that the default operation after wiping a resource will keep the system in the stable region long enough for the particular resource to recover normal operations.

% Also, a paper on safe restarts\cite{abdi2017application}.


