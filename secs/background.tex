\subsection{Modeling Networked CPS}

The continuous-time plant in presence of noise and attack on output:
\begin{align}
    \dot{x}_p &= f_p(x_p, \hat{u}) + w_p \qquad y = g_p(x_p, \eta (t)) + v_p 
\end{align}

where, $x_p \in \mathbb{R}^{n_p}$ represents the state of the plant, $\hat{u} \in \mathbb{R}^{n_u}$ represents the control values implemented at the plant, $w_p \sim N(0, Q)$ is the additive zero mean Gaussian noise, $y \in \mathbb{R}^{n_y}$ is the output of the plant, $v_p \sim N(0, R)$ is the additive zero mean Gaussian measurement noise, and $\eta (t)$ is the variable for attack on sensor. During attack, $\eta (t)$ corrupts the output of the plant ($y$), which is measured by sensors. 


The plant is controlled by a controller over the shared communication network, whose equations are:
\begin{align}
    \dot{x}_c &= f_c(x_c, \hat{y}) \qquad u = g_c(x_c, \hat{y}, \mu (t)) 
\end{align}

where, $x_c \in \mathbb{R}^{n_c}$ represents the state of the controller,  $\hat{y} \in \mathbb{R}^{n_y}$ is the most recent output (either good or corrupted by attack) of the plant available to the controller, $u \in \mathbb{R}^{n_u}$ represents the controller output, and $\mu (t)$ is the variable for attack on actuator. The controller output ($u$) can be manipulated by cyber or physical attack, which is represented by $\mu (t)$. Due to the presence of communication network, $u \neq \hat{u}$ and $y \neq \hat{y}$. In this setup, sensors are time-driven and both actuator and controller are event-driven. As the controllers, sensors, and actuators are connected via a shared network, they are subject to varying transmission intervals and varying delays. Due to varying transmission intervals, the instants ($t_k \in \mathbb{R}_{\geq 0}, \; k \in \mathbb{N}$) at which the plant outputs and control values are sampled and transmitted over network are non-equidistantly spaced in time. These transmitted values are received by other units in the network after a delay of $\tau_k \in \mathbb{R}_{\geq 0}$, with $\tau_k \in [\tau_{min}, \tau_{max}]$, for all $k \in \mathbb{N}$. This delay is due to the speed at which data travels in the network and it should be less than transmission interval to ensure correct operation of the controllers. In NCS, a scheduling protocol is used in the network to ensure data from all sensors and actuators are not transmitted at the same time. Moreover, while updating the values of $\hat{y}$ and $\hat{u}$, the network is assumed to operate in a zero-order-hold (ZOH) manner i.e. these values remain constant while being updated. 


TODO - find a formal model to embed the above equations into some logical formula that can be checked (with wieghted max-smt?).
We want to know, given a single scheduler (instance of the model), how much delay must be introduced to inalidate the saftey/stability specifications of the system. If the attacker can only introduce delay at one point (step) in the system, how much delay must be introduced. If the attacker can introduce a global delay, how much is needed? What about subsets of modules (steps in system).